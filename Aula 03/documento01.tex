\documentclass[12pt, a4paper]{article}
\usepackage[brazilian]{babel}% Traduz o documento para pt-br
\usepackage[utf8]{inputenc}% Pacote para reconhecimento de acentuação
\usepackage{lipsum}% Texto aleatório

\newtheorem{teorema}{Teorema}[section]
\newtheorem{definicao}[teorema]{Definição}
\newtheorem{proposicao}{Proposição}[section]

\begin{document}
	
\section{Primeira parte}

\begin{teorema}
	Todo quadrado tem quatro angulos retos.
\end{teorema}

\begin{definicao}
	Um quadrado é um polígono de quatro lados.
\end{definicao}

\begin{teorema}
	Todo triangulo tem três lados.
\end{teorema}

\begin{proposicao}
	Um pentágono tem 5 lados.
\end{proposicao}

\begin{teorema}[Pentágono]
	Todo quadrado tem quatro angulos retos.
\end{teorema}

\section{Segunda parte}

\begin{teorema}
	Todo triangulo tem três lados.
\end{teorema}

\begin{definicao}
	Um quadrado é um polígono de quatro lados.
\end{definicao}


\end{document}